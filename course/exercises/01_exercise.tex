\documentclass[a4paper,12pt]{article}

\usepackage{vmargin}
\setpapersize[portrait]{A4}
%% \setmarginsrb{25mm}{10mm}{20mm}{20mm}% left, top, right, bottom
%% {12pt}{15mm}% head heigth / separation
%% {0pt}{15mm}% bottom height / separation
\setmargnohfrb{20mm}{20mm}{20mm}{20mm}

\setlength{\parindent}{0mm}
\setlength{\parskip}{\medskipamount}

\usepackage[T1]{fontenc}
\usepackage[utf8]{inputenc}
\usepackage[english]{babel}

\usepackage[sc]{mathpazo}  % Adobe Palatino + Courier
\usepackage{courier}
\usepackage{textcomp} 


\usepackage{amsmath,amssymb,amsthm}
\usepackage{graphicx,rotating}
\usepackage{color}
\usepackage{array,hhline,booktabs}
\usepackage{xspace}
\usepackage{url}
\usepackage{alltt}

%% same markup for R example code as in slides
\definecolor{osnared}{rgb}{.65,.05,.10}
\definecolor{osnablue}{rgb}{.03, .09, .56} 
\newcommand{\REM}[2][\small]{\textsf{#1\color{osnared}\# #2}}
\newenvironment{Rout}[1][\small]{%
  \begin{normalsize}#1\color{osnablue}\bfseries}{%
  \color{black}\mdseries\end{normalsize}}


\begin{document}

\emph{Marco Baroni \& Stefan Evert} \hfill %
{\small \url{http://SIGIL.R-Forge.R-Project.org/}}

\begin{center}
  \textbf{\large Statistical Analysis of Corpus Data with R}

  \textbf{\large --- Exercise Sheet for Unit \#1 ---}
\end{center}

In this first exercise, you will familiarise yourself with basic R
operations, reading the R documentation, and working with data frames, which
are the fundamental statistical data structure used by R.  For this purpose,
the \texttt{SIGIL} package contains a data frame with complete metadata 
information for all 4,048 texts of the \emph{British National Corpus}.%
\footnote{See \url{http://www.natcorp.ox.ac.uk/} for more information about
  the BNC and available metadata.} %
You should be able to access the BNC metadata under the name \texttt{BNCmeta} after loading the
package:

\begin{alltt}
    > library(SIGIL)
    > nrow(BNCmeta)    \REM{check that metadata table is available}\begin{Rout}
    [1] 4048 \end{Rout}
    > attach(BNCmeta)  \REM{for convenient access to table columns}
\end{alltt}

Here are some things to do with the metadata table.  You should try yourself
on all these tasks for full credit. Notice that not all the commands are in
the course slides: explore the documentation and other R teaching materials.

\begin{enumerate}
\item How many rows and columns does the metadata table have?  How many
  meta-information variables are annotated?  Read the help page for the data set.
\item How many different genres are there in the BNC?  Which genre contains
  the smallest number of texts?  (Hint: use \texttt{table()} to obtain
  frequency counts for the levels of a factor variable.)
  %% > sort(table(genre))
  %% fiction/drama (2), lectures/commerce (3), essay/university (3)
\item Save the metadata table to a \texttt{.csv} file (recall that CSV stands
  for \emph{comma-separated values}), which you can load into a spreadsheet
  application such as Microsoft Excel or OpenOffice.org Calc.  Use the
  spreadsheet to explore the metadata tables.
\item Summarise the distribution of text lengths (measured either in number of
  words \verb|n_words| or number of sentences \verb|n_s|).  The text type
  \emph{written-to-be-spoken} contains some ``outliers'', i.e.\ texts with
  rather unusual lengths.  Use \texttt{boxplot()} to identify these outliers.  
  %% > boxplot(n_words[text_type=="written-to-be-spoken"])
  Can you list the titles of the outlier texts?
  %% > title[ text_type == "written-to-be-spoken" & n_words <= 20000 ]
  %% [1] [Miscellaneous prayers]. Sample containing about 7193 words of material written to be spoken (domain: belief and thought)
  %% [2] [Sermons]. Sample containing about 14430 words of material written to be spoken (domain: belief and thought)             
  %% [3] Speeches by Tony Hall. Sample containing about 8137 words of material written to be spoken (domain: social science)      
  %% [4] Scottish TV -- sports news scripts. Sample containing about 2686 words of material written to be spoken (domain: leisure)
  %% [5] [Central television news scripts]. Sample containing about 6789 words of material written to be spoken (domain: leisure) 
\item Do text lengths differ between text types?  Do they differ between male
  and female authors?  (You can use the ``formula'' interface to
  \texttt{boxplot()} for this task, but you will have to remove outliers first
  or rescale the plot.)
  %% > boxplot(n_words ~ text_type, ylim=c(0,100000))
\item Later in the course, we might perform machine-learning experiments to
  distinguish between male and female authors.  Produce a subset of the
  metadata table containing only texts for which author sex is known, omitting
  the title and irrelevant metadata columns (esp.\ those which have only a
  single value in the subset).
  %% > B2 <- subset(BNC, author_sex == "male" | author_sex == "female")
  %% > summary(B2)  # to identify irrelevant variables, then select in subset() command
\item The number of words and number of sentences as measures of text length
  should be highly correlated.  Illustrate this correlation with a suitable
  plot.  Compute the correlation coefficient between these two variables and
  its 95\% confidence interval.
  Judging from the plot, do you think there is a simple linear relationship?
  %% > plot(n_words, n_s, xlim=c(0,100000), ylim=c(0,10000))
  %% > cor.test(n_words, n_s)

\end{enumerate}



\end{document}
