\documentclass[a4paper,12pt]{article}

\usepackage{vmargin}
\setpapersize[portrait]{A4}
%% \setmarginsrb{25mm}{10mm}{20mm}{20mm}% left, top, right, bottom
%% {12pt}{15mm}% head heigth / separation
%% {0pt}{15mm}% bottom height / separation
\setmargnohfrb{20mm}{20mm}{20mm}{20mm}

\setlength{\parindent}{0mm}
\setlength{\parskip}{\medskipamount}

\usepackage[english]{babel}
\usepackage[latin1]{inputenc}

\usepackage[T1]{fontenc}
\usepackage{textcomp}  % this can break some outline symbols in CM fonts, use only if absolutely necessary

%% \usepackage{stefan-fonts}  % commercial Charter fonts with full math support

% \usepackage{mathptmx}  % use Adobe Times as standard font with simulated math support
\usepackage[sc]{mathpazo}  % use Adobe Palatino as standard font with simulated math support
\usepackage{courier}

%% \usepackage{pifont}
%% \usepackage{eucal}

\usepackage{amsmath,amssymb,amsthm}
\usepackage{graphicx,rotating}
\usepackage{color}
\usepackage{array,hhline,booktabs}
\usepackage{xspace}
\usepackage{url}
\usepackage{alltt}
%% \usepackage{ifthen,calc,hyphenat}
%% \usepackage{pgf,pgfarrows,pgfnodes,pgfautomata,pgfheaps,pgfshade,xcolor,colortbl}

\newcommand{\REM}[1]{\textrm{\color[rgb]{.7,.2,.1}\# #1}}

\begin{document}

\emph{Marco Baroni \& Stefan Evert} \hfill %
{\small \url{http://purl.org/stefan.evert/SIGIL}}

\begin{center}
  \textbf{\Large Statistical Analysis of Corpus Data with R}

  \textbf{\large Exercise Sheet \#3}
\end{center}

\emph{Multi-dimensional scaling} (MDS) is another popular
dimensionality reduction technique (see, e.g., Cox and Cox, 2001,
\emph{Multidimensional Scaling}, Chapman \& Hall).

Perform an MDS analysis of the Italian NN compound data, based on
(scaled versions) of the cues described in the course slides.

\begin{enumerate}
\item MDS operates on a \emph{distance} matrix, a symmetric matrix of
  distances between each point in the data-set and each other
  point. Thus, the first thing you will need to do is to generate a
  distance matrix from the cue matrix. Look at the documentation for
  the \texttt{dist()} function, and use it to generate distance
  matrices using two different methods to compute distance.
\item In order to perform MDS, you will use the \texttt{cmdscale()}
  function: take a look at its documentation, and run MDS on each of
  your distance matrices (if you want to perform some further
  exploration of MDS, consider also the \texttt{sammon()} and
  \texttt{isoMDS()} functions in the venerable \texttt{MASS} package).
\item Plot the compounds in the first two dimensions produced by the
  MDS analyses, using different colours for relational and attributive
  compounds.
\item Try \emph{k-means} clustering on the MDS outputs, and look at
  performance by cross-tabulating the clusters and the
  relational/attributive labels.
\end{enumerate}






% \begin{frame}[fragile]
%   \frametitle{Appendix: Multi-Dimensional Scaling}
%   \begin{alltt}
% mds<-cmdscale(dist(scaled))
% plot(mds,type="n")
% points(mds[TYPE=="re",],col="blue")
% points(mds[TYPE=="at",],col="red")

% \REM{see also sammon, isoMDS in the
% # MASS package}
%   \end{alltt}
% \end{frame}

\end{document}
