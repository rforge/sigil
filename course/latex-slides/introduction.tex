\documentclass{beamer}
%\documentclass[handout]{beamer}

\usetheme{default}

\usepackage[english]{babel}
\usepackage[latin1]{inputenc}

\usepackage{times}
\usepackage[T1]{fontenc}

%%
%%  some useful macro definitions & configuration settings
%%

\usepackage{latexsym}
\usepackage{graphicx}
\usepackage{url}
\usepackage{alltt}              % code examples with nicely formatted comments
\usepackage{xcolor}
\usepackage{pifont}
\usepackage{xspace}
\usepackage{array,booktabs}

\definecolor{quietred}{rgb}{.6,.2,.1}
\definecolor{quietblue}{rgb}{.1,.2,.7}
\definecolor{brightred}{rgb}{.9,0,.1}

%% > plot(x,y)      \REM{this produces a scatterplot}
\newcommand{\REM}[1]{\textsf{\small\color{quietred}\# #1}}

%% nice colour for R output: \begin{Rout} .. \end{Rout}
\newenvironment{Rout}{%
  \begin{footnotesize}\color{quietblue}\bfseries}{%
  \color{black}\mdseries\end{footnotesize}}

%% ... This is something \h{important}. ...
\newcommand<>{\h}[1]{\textbf#2{\color#2{quietred}#1}}
\newcommand<>{\hh}[1]{\textbf#2{\color#2{brightred}#1}}

%% cite \textcite{some text} in roman italic font
\newcommand{\textcite}[1]{\textrm{\textit{#1}}}

%% how can you live without the arrow (\so) and the hand (\hand) ?
\newcommand{\so}{\ding{234}\xspace}
\newcommand{\So}{\hh{\ding{229}}\xspace}
\newcommand{\hand}{\ding{43}\xspace}

%% \p{X=k};  \pC{X=k}{Y=l};  \bigp{X_i = k};   \pscale{\frac{Z}{S^2}};
%% probability P(X=k) and conditional probability P(X=k|Y=l), also with larger or scaled parentheses
%% \p[\theta]{X=k};  \pC[\text{interpolated}]{X=k}{Y=l};  ...
%% with optional subscripts (for model probability, null probability, etc.)
\newcommand{\p}[2][]{\mathop{\text{Pr}_{#1}}(#2)}
\newcommand{\pscale}[2][]{\mathop{\text{Pr}_{#1}}\!\left(#2\right)}
\newcommand{\bigp}[2][]{\mathop{\text{Pr}_{#1}}\bigl(#2\bigr)}
\newcommand{\pC}[3][]{\p[#1]{#2\,|\,#3}} 
\newcommand{\pCscale}[3][]{\pscale[#1]{#2\left|\,#3\right.\!}} 
\newcommand{\bigpC}[3][]{\bigp[#1]{#2\bigm|#3}} 

%% \Exp{X};  \Var{X};  \Exp[0]{X};  \Var[0]{X};  
%% \bigExp{X}; \bigVar{X}; \Expscale{X};  \Varscale{X};
%% expectation E[X] and variance V[X], expectation and variance under null hypothesis, 
%% and variants with largeer or scaled brackets
\newcommand{\Exp}[2][]{\text{E}_{#1}[#2]}
\newcommand{\Var}[2][]{\mathop{\text{Var}}_{#1}[#2]}
\newcommand{\bigExp}[2][]{\text{E}_{#1}\!\bigl[#2\bigr]}
\newcommand{\bigVar}[2][]{\mathop{\text{Var}}_{#1}\bigl[#2\bigr]}
\newcommand{\Expscale}[2][]{\text{E}_{#1}\left[#2\right]}
\newcommand{\Varscale}[2][]{\mathop{\text{Var}}_{#1}\left[#2\right]}


\title[R for Computational Linguists]{Statistical Analysis of Corpus Data with R}
\subtitle{A Gentle Introduction\\ for Computational Linguists and Similar Creatures}

\author[Baroni \& Evert]{Designed by Marco Baroni\inst{1} and Stefan Evert\inst{2}}
\institute{
  \inst{1}Center for Mind/Brain Sciences (CIMeC)\\
  University of Trento
  \and
  \inst{2}Institute of Cognitive Science (IKW)\\
  University of Onsabr�ck
}
\date{}


\begin{document}

\frame{\titlepage}


 \begin{frame}
   \frametitle{Outline}
   \tableofcontents
 \end{frame}

\section{General Information}

\subsection{What is R?}

\begin{frame}
  \frametitle{Why do we need statistics?}

  \begin{itemize}
  \item<2-> \h{Significance} (control for sampling variation)
    \begin{itemize}
    \item all linguistic data are samples (of language, speakers, \ldots)
    \item observed effects may be coincidence of particular sample
    \item[\So] \textbf{inferential statistics}
    \item[]
    \end{itemize}
  \item<3-> Managing \h{large data sets}
    \begin{itemize}
    \item statistical summaries, data analysis, visualisation
    \item e.g.\ collocations as compact summary of word usage
    \item[\So] \textbf{descriptive statistics}
    \item[]
    \end{itemize}
  \item<4-> Discovering \h{latent} (hidden) \h{properties}
    \begin{itemize}
    \item clustering, multivariate analysis, distributional semantics
    \item advanced statistical modelling (e.g.\ mixed-effects models)
    \item[\So] \textbf{exploratory data analysis}
    \end{itemize}
  \end{itemize}
\end{frame}

\begin{frame}
  \frametitle{\textbf{R} -- An environment for statistical programming}

  \begin{itemize}
  \item<1-> ``Traditional'' statistical software packages offer specialised
    procedures (e.g.\ SAS) or interactive GUI (e.g.\ SPSS)
    \begin{itemize}
    \item[]
    \end{itemize}
  \item<2-> New approach: statistical programming language \hh{S} with interactive
    environment (Bell Labs, since 1976)
    \begin{itemize}
    \item \emph{White Book} (version 3, 1992); \emph{Green Book} (version
      4, 1998)
    \item commercial: S-Plus (Insightful Corporation, since 1987)
    \item[]
    \end{itemize}
  \item<3-> \hh{R} is an open-source implementation of the S language
    \begin{itemize}
    \item originally by Ross Ihaka and Robert Gentleman (Auckland)
    \item open-source development since mid-1997
    \end{itemize}
  \end{itemize}
\end{frame}

\begin{frame}
  \frametitle{\textbf{R} -- An environment for statistical programming}
  
  \vspace{5mm}
  \begin{columns}
    \begin{column}{35mm}
      \includegraphics[width=4cm]{img/Rlogo}
    \end{column}
    \begin{column}{7cm}
      \begin{itemize}
      \item binary packages available for Linux, Mac OS X and Windows
      \item 64-bit versions on Linux and OS X
      \item extensive documentation \& tutorials
      \item hundreds of add-on packages ready to install from CRAN
      \end{itemize}
    \end{column}
  \end{columns}
  \vspace{8mm}
  \begin{center}
    \h{\LARGE\url{http://www.R-project.org/}}
    
    \vspace{6mm}
    Recommended Windows GUI:\\
    \h{Tinn-R} from \url{http://www.sciviews.org/}
  \end{center}
\end{frame}

\begin{frame}
  \frametitle{More about \textbf{R}}
  
  \begin{itemize}
  \item Advantages of R
    \begin{itemize}
    \item free \& open source
    \item many add-on packages with state-of-the-art algorithms
    \item large, enthusiastic and helpful user community
    \item easy to automate and extend (every analysis is a program)
    \item no point \& click interface
    \item[]
    \end{itemize}
  \item<2-> Disadvantages
    \begin{itemize}
    \item learning curve sometimes rather steep
    \item not good at manipulating non-English text (yet)
    \item no built-in data editor (spreadsheet)
    \item no point \& click interface
    \end{itemize}
  \end{itemize}
\end{frame}

\subsection{About this course}

\begin{frame}
  \frametitle{Goals of the course}

  \begin{itemize}
  \item Learn R basics and elementary R programming
  \item Get to know R implementations of statistical techniques, data analysis
    and visualisation that are useful in various areas of (computational)
    linguistics
  \item A little bit of background in the statistical analysis of corpus
    frequency data along the way
  \item Practice your R skills on real-life data-sets
  \end{itemize}

\end{frame}

\begin{frame}
  \frametitle{What this course is \emph{not} about}

  \begin{itemize}
  \item Theoretical foundations of statistics
  \item Specific statistical methods
  \item Cookbook recipes for particular analyses with R
  \end{itemize}

\end{frame}

\begin{frame}%% <beamer:0| handout:0> %% TEMP
  \frametitle{What you should know}

  \begin{itemize}
  \item Very basic math and statistics\\ (vectors, logarithms,
    correlation, $t$-tests, \ldots)
  \item Some familiarity with programming/scripting\\
    and/or with a command-line environment
  \item Interest in (computational) linguistics
  \end{itemize}

\end{frame}


\begin{frame}%% <beamer:0| handout:0> %% TEMP
  \frametitle{Course syllabus}
  
  \begin{itemize}
  \item Introduction to R: set-up, data manipulation and exploration,
    plotting, basic statistics, input/output
  \item Hypothesis tests for corpus frequency data
  \item Using an R extension package:\\ 
    modelling word frequency distributions with zipfR
  \item Unsupervised multivariate data exploration:\\
    principal component analysis and clustering
  \item Co-occurrence statistics and frequency comparisons:
    contingency tables, association measures, evaluation
  \item Efficient data processing using vector operations
  \item The limitations of random sampling models for corpus data
  \end{itemize}
\end{frame}

%%% Saarbruecken 2009
% \begin{frame}
%   \frametitle{Course schedule}

%   \h{Wednesday}
%   \begin{itemize}
%   \item Introduction: installing R, data manipulation, input/output
%   \item Practice: your first steps in R
%   \item Hypothesis tests for frequency data (\so inferential)
%   \item Practice: frequency comparisons on CroCo
%   \item[]
%   \end{itemize}

%   \h{Thursday}
%   \begin{itemize}
%   \item Clustering \& multivariate data analysis (\so exploratory)
%   \item Practice: clustering texts from CroCo
%   \item Collocations -- theory \& implementation (\so descriptive)
%   \item Practice: large-scale collocation analysis in R
%   \end{itemize}
% \end{frame}

\begin{frame}<beamer:1| handout:0>
  \frametitle{Introductions}

  \begin{center}
    \includegraphics[height=6cm]{img/uncle-sam-wants-you}

    \vspace{5mm}
    \begin{huge}
      \hh{Who are you?}
    \end{huge}
  \end{center}
\end{frame}

\begin{frame}
  \frametitle{R textbooks for (computational) linguists}
  \framesubtitle{Much more comprehensive theoretical background and cookbook examples}

  \begin{itemize}
  \item Stefan Th.\ Gries (to appear). \h{\emph{Statistics for Lingustics with R:
        A practical introduction}}. Mouton de Gruyter.
    \begin{itemize}
    \item German original is already available
    \item[]
    \end{itemize}
  \item Shravan Vasishth (2006--2009). \h{\emph{The foundations of statistics: A
        simulation-based approach}}.
    \begin{itemize}
    \item http://www.ling.uni-potsdam.de/\textasciitilde{}vasishth/SFLS.html
    \item[]
    \end{itemize}
  \item R.~Harald Baayen (2008). \h{\emph{Analyzing Linguistic Data: A practical
        introduction to statistics}}. CUP.
    \begin{itemize}
    \item http://www.ualberta.ca/\textasciitilde{}baayen/publications.html
    \item if you download the PDF, you should also buy the book
    \end{itemize}
  \end{itemize}
\end{frame}

\begin{frame}
  \frametitle{Other recommended textbooks on statistics and R}

  \begin{itemize}
  \item Peter Dalgaard (2008). \h{\emph{Introductory Statistics with R}}, 2nd ed. New York:
    Springer.
    \begin{itemize}
    \item[]
    \end{itemize}
  \item Morris H.\ DeGroot and Mark J.\ Schervish (2002). \h{\emph{Probability and
      Statistics}}, 3rd ed. Addison Wesley.
    \begin{itemize}
    \item Stefan's favourite statistics textbook
    \item[]
    \end{itemize}
  \item John M.\ Chambers (2008). \h{\emph{Software for Data Analysis:
        Programming with R}}. New York: Springer.
    \begin{itemize}
    \item[]
    \end{itemize}
  \item Christopher Butler (1985), \h{\emph{Statistics in Linguistics}}. Oxford:
    Blackwell.
    \begin{itemize}
    \item out of print and available online for free download
    \item \url{http://www.uwe.ac.uk/hlss/llas/statistics-in-linguistics/bkindex.shtml}
    \end{itemize}
  \end{itemize}
\end{frame}

\begin{frame}
  \frametitle{Course materials}

  \begin{itemize}
  \item Handouts, example scripts and data sets are available on our homepage
    for this course:\\[4mm]
    \begin{center}\color{blue}
      \textbf{\url{http://purl.org/stefan.evert/SIGIL/}}
    \end{center}
  \item[]
  \item You will also find additional material, software and\\
    links to background reading there
  \end{itemize}
\end{frame}

%%%%%%%%%%%%%%%%%%%%%%%%%%%%%%%%%%%%%%%%%%%%%%%%%%%%%%%%%%%%%%%%%%%%%%%%

\AtBeginSection[]
{
  \begin{frame}
    \frametitle{Outline}
    \tableofcontents[current,currentsection]
  \end{frame}
}


\AtBeginSubsection[]
{
  \begin{frame}
    \frametitle{Outline}
    \tableofcontents[current,currentsubsection]
  \end{frame}
}

\section{R Basics}

% \begin{frame}
%   \frametitle{R}

%   \begin{itemize}
%   \item \url{http://www.r-project.org/}
%   \item Free, open-source implementation of the S language developed by
%     Venables and Ripley
%   \item Available for Linux/Unix, Mac and Windows
%   \item Command-line interface and GUI (for Mac and Windows)
%     \begin{itemize}
%     \item for Windows, we recommend \url{www.sciviews.org} GUI
%     \end{itemize}
%   \item Non-interactive use possible via scripting
%   \item Less user-friendly than other statistical software,\\
%     but immensely more powerful
%   \item Offers a wealth of extension packages implementing an impressive range
%     of classic and cutting edge statistical\\ and data analysis techniques
%   \end{itemize}
% \end{frame}

\subsection{Basic functionalities}

\begin{frame}[fragile]
  \frametitle{R as an oversized calculator}
  
\begin{alltt}
> 1+1
[1] 2

> a <- 2     \REM{assignment does \emph{not} print anything by default}

> a * 2
[1] 4

> log(a)     \REM{natural, i.e.\ base-\(e\) logarithm}
[1] 0.6931472

> log(a,2)   \REM{base-2 logarithm}
[1] 1
\end{alltt}

\end{frame}



\begin{frame}[fragile]
  \frametitle{Basic session management}
  \framesubtitle{Some of it is not necessary if you only use the GUI}
  
\begin{alltt}
\REM{to start R on command line, simply type \texttt{\textbf{R}}}

setwd("path/to/data")  \REM{or use GUI menus}

ls()                   \REM{probably empty for now}

ls                     \REM{notice difference with previous line}

quit()                 \REM{or use GUI menus}
quit(save="yes")
quit(save="no")

\REM{NB: at least some interfaces support history recall, tab completion}
\end{alltt}

\end{frame}


\begin{frame}[fragile]
  \frametitle{Vectorial math}

\begin{alltt}
> a <- c(1,2,3) \REM{\texttt{\textbf{c}} (for \emph{combine}) creates vectors}

> a * 2   \REM{operators are applied to each element of a vector}
[1] 2 4 6

> log(a)  \REM{also works for most standard functions}
[1] 0.0000000 0.6931472 1.0986123

> sum(a)  \REM{basic vector operations: sum, length, product, \ldots}
[1] 6

> length(a)
[1] 3

> sum(a)/length(a)
[1] 2
\end{alltt}

\end{frame}


\begin{frame}[fragile]
  \frametitle{Initializing vectors}

\begin{alltt}
> a <- 1:100            \REM{integer sequence}
> a

> a <- 10^(1:100)

> a <- seq(from=0, to=10, by=0.1) \REM{general sequence}

> a <- rnorm(100)       \REM{100 random numbers}

> a <- runif(100, 0, 5) \REM{what you're used to from Java etc.}
\end{alltt}
\end{frame}

\begin{frame}[fragile]
  \frametitle{Summary statistics}

\begin{alltt}
> length(a)

> summary(a)  \REM{statistical summary of numeric vector} \begin{Rout}
   Min. 1st Qu.  Median    Mean 3rd Qu.    Max. 
0.02717 0.51770 1.05200 1.74300 2.32600 9.11100  \end{Rout}

> mean(a)

> median(a)

> sd(a)       \REM{standard deviation is not included in summary}

> quantile(a) \begin{Rout}
    0%    25%    50%    75%   100% 
0.0272 0.5177 1.0518 2.3261 9.1107 \end{Rout}

> quantile(a,.75)
\end{alltt}

\end{frame}


\begin{frame}[fragile]
  \frametitle{Basic plotting}

\begin{alltt}
> a<-2^(1:100)        \REM{don't forget the parentheses!}
> plot(a)

> x<-1:100            \REM{most often: plot \(x\) against \(y\)}
> plot(x,a)

> plot(x,a,log="y")   \REM{various logarithmic plots}
> plot(x,a,log="x")
> plot(x,a,log="xy")
> plot(log(x),log(a))

> hist(rnorm(100))    \REM{histogram and density estimation}
> hist(rnorm(1000))
> plot(density(rnorm(100000)))
\end{alltt}

\end{frame}

\begin{frame}[fragile]
  \frametitle{(Slightly less) basic plotting}

\begin{alltt}
> a <- rbinom(10000,100,.5)
> hist(a)

> hist(a, probability=TRUE)
> lines(density(a))

> hist(a, probability=TRUE)
> lines(density(a), col="red", lwd=3)

> hist(a, probability=TRUE, 
  main="Some Distribution", xlab="value",
  ylab="probability")
\REM{better to type command on a single line!}
> lines(density(a), col="red", lwd=3)
\end{alltt}

\end{frame}

\begin{frame}[fragile]
  \frametitle{Help!}

\begin{alltt}
> help("hist")  \REM{R has excellent online documentation}
> ?hist         \REM{short, convenient form of the help command} 

> help.search("histogram")

> ?help.search

> help.start()  \REM{searchable HTML documentation}

\REM{or use GUI menus to access \& search documentation}
\end{alltt}

\end{frame}


\begin{frame}
  \frametitle{Installing add-on packages}

  \begin{itemize}
  \item Much of R's power comes from its add-on packages
  \item Can be downloaded from CRAN with GUI installer
    \begin{itemize}
    \item automatically installs other required packages
    \item Mac OS X: check ``install dependencies''
    \item Windows: only most essential dependencies installed
    \item[]\pause
    \end{itemize}
  \item The ``sumo'' package for linguists: \h{\texttt{languageR}}
    \begin{itemize}
    \item data sets \& utilities for Baayen (2008)
    \item also installs most other packages that you'll need
    \end{itemize}
  \item Magic command:
    \texttt{install.packages("languageR",\\ .libPaths()[1],
      dependencies=TRUE)}
    \begin{itemize}
    \item[]\pause
    \end{itemize}
  \item Other highly recommended packages:
    \begin{itemize}
    \item \texttt{corpora} for a few data sets used in this course
    \item \texttt{rgl} and \texttt{misc3d} for interactive 3D graphics
    \item \texttt{plyr} and \texttt{gsubfn} for convenience
    \item advanced: \texttt{rggobi} for high-dimensional visualisation
    \end{itemize}
  \end{itemize}
\end{frame}


\begin{frame}[fragile]
  \frametitle{Your first R script}

  \begin{itemize}
  \item Simply type R commands into a text file \& save it
  \item Use built-in GUI functionality or external text editor
    \begin{itemize}
    \item Microsoft Word is \emph{not} a text editor!
    \item nor is Apple's TextEdit application \ldots
    \end{itemize}
  \item[]
  \item Execute R script from GUI editor or by typing
    \begin{alltt}
> source("my_script.R") \REM{more about files later}
> source(file.choose()) \REM{select with file dialog box}
    \end{alltt}
  \item Just typing a variable name will not automatically print its value in
    a script: use \verb_print(sd(a))_ instead of \verb_sd(a)_
  \end{itemize}
\end{frame}

\subsection{External files and data-frames}

\begin{frame}[fragile]
  \frametitle{Input from an external file}

  \begin{itemize}
  \item We like to keep our data in space- or TAB-delimited text files with a
    first row (``header'') labeling the fields, like so:
\begin{verbatim}
word  frequency cat
dog   15        noun
bark  10        verb 
\end{verbatim}
  \item This is an easy format to import into R, and it is easy to
    convert from/to other tabular formats using standard tools
  \item We assume that external input is always in this format\\ (or can
    easily be converted to it)
    \begin{itemize}
    \item spreadsheet applications prefer CSV format (comma-separated values)
    \item Microsoft Excel is a nice table editor,\\ but beware of localised
      number formats
    \end{itemize}
  \end{itemize}

\end{frame}



\begin{frame}[fragile]
  \frametitle{Reading a TAB-delimited file with header}
  
\begin{alltt}
> brown <- read.table("brown.stats.txt",
  header=TRUE)
\REM{if file is not in working directory, you must specify the full path}
\REM{(or use \texttt{setwd()} function we introduced before)}

\REM{exact behaviour of \texttt{file.choose()} depends on operating system}
> brown <- read.table(file.choose(), header=TRUE)

\REM{more robust if you are sure file is in tab-delimited format}
> brown <- read.delim("brown.stats.txt")
\end{alltt}

\end{frame}


\begin{frame}[fragile]
  \frametitle{Reading and writing CSV files}
  
\begin{alltt}
\REM{R can also read and write files in CSV format}
> write.csv(brown, "brown.stats.csv",
  row.names=FALSE)
\REM{this is convenient for exchanging data with database and}
\REM{spreadsheet software (or using Excel as a data editor)}

\REM{NB: comma-separated values are not always separated by commas}
\REM{(e.g.\ in German; use \texttt{write.csv2} if Excel doesn't recognise columns)}
> write.csv2(brown, "brown.stats.csv",
  row.names=FALSE)

\REM{TASK: load \texttt{brown.stats.csv} into Excel or OpenOffice.org}

\REM{check generated CSV file (use \texttt{read.csv2} with \texttt{write.csv2} above)}
> brown.csv <- read.csv("brown.stats.csv")
> all.equal(brown.csv, brown)
\end{alltt}

\end{frame}


\begin{frame}
  \frametitle{Data-frames}

  \begin{itemize}
  \item The commands above create a \h{data frame}
  \item This is the basic data structure (object)\\
    used to represent statistical tables in R
    \begin{itemize}
    \item rows = objects or ``observations''
    \item columns = variables, i.e.\ measured quantities
    \item[]
    \end{itemize}
  \item Different types of variables
    \begin{itemize}
    \item numerical variables (what we've used so far)
    \item Boolean variables
    \item factor variables (nominal or ordinal classification)
    \item string variables
    \item[]
    \end{itemize}
  \item Technically, data frames are collections of column vectors (of the same
    length), and we will think of them as such
  \end{itemize}

\end{frame}


\begin{frame}[fragile]
  \frametitle{Data-frames}

\begin{alltt}
> summary(brown)

> colnames(brown)

> dim(brown)       \REM{number of rows and columns}

> head(brown)

> plot(brown)
\end{alltt}

\end{frame}


\begin{frame}[fragile]
  \frametitle{Access vectors inside a data frame}

\begin{alltt}
> brown$to

> head(brown$to)

\REM{TASK: compute summary statistics (length, mean, max, etc.)}
\REM{for vectors in the Brown data frame}

\REM{what does the following do?}
> summary(brown$ty / brown$to)

> attach(brown)   \REM{attach data frame for convenient access}
> summary(ty/to)
> detach()  \REM{better to detach before you attach another frame}
\end{alltt}

\end{frame}

\begin{frame}[fragile]
  \frametitle{More data access}

\begin{alltt}
> brown$ty[1]    \REM{vector indexing starts with 1}
> brown[1,2]     \REM{row, column}

> brown$ty[1:10] \REM{use arbitrary vectors as indices}
> brown[1:10,2]

> brown[1,]
> brown[,2]
\end{alltt}

\end{frame}


\begin{frame}[fragile]
  \frametitle{Conditional selection}


\begin{alltt}
> brown[brown$to < 2200, ]  \REM{index with Boolean vector}
> length(brown$ty[brown$to >= 2200])
> sum(brown$to >= 2200)    \REM{standard way to count matches}

> subset(brown, to < 2200)  \REM{no need to attach here}
> lessdata <- subset(brown, to < 2200)

> a <- brown$ty[brown$to >= 2200]

\REM{equality: == (also works for strings)}
\REM{inequality: !=}
\REM{complex constraints: and &, or |, not !}
\REM{NB: always use single characters, not && or ||}
\end{alltt}

\end{frame}


\subsection{A simple case study: comparing Brown and LOB documents}

\begin{frame}
  \frametitle{Type, token and word length counts 
    \\in the Brown and LOB documents}

  Variables:
  \begin{description}
  \item[to] Token count
  \item[ty] Type count (\emph{distinct} words)
  \item[se] Sentence count
  \item[towl] Average word length\\ (averaged across tokens in document)
  \item[tywl] Average word length\\ (averaged across distinct types in
    document)
  \end{description}
\end{frame}

\begin{frame}
  \frametitle{Procedure}

    \begin{itemize}
    \item Collect basic summary statistics for the two corpora
    \item Check if there is a significant difference in the token counts
      (since document length was controlled by corpus builders)
    \item If difference is significant (we will see that it is), then
      type counts are not directly comparable, and sentence
      counts should be normalized (divide by token count)
    \item Is word length correlated to document length? (in which
      case, corpus comparison would also not be appropriate)
      \pause
    \item Please read the LOB data set into a data frame named \texttt{lob}
      now, and take a look at its basic statistics
    \item Also, plot the data frame for a first impression of correlations
      between the variables
    \end{itemize}

\end{frame}

\begin{frame}[fragile]
  \frametitle{Comparing token counts}

\begin{alltt}
> boxplot(brown$to,lob$to)
> boxplot(brown$to,lob$to,names=c("brown","lob"))
> boxplot(brown$to,lob$to,names=c("brown","lob"),
  ylim=c(1500,3000))
> ?boxplot

> t.test(brown$to, lob$to)
> wilcox.test(brown$to, lob$to)

> brown.to.center <- brown$to[brown$to > 2200 
  & brown$to < 2400]
> lob.to.center <- lob$to[lob$to > 2200 
  & lob$to < 2400]

> t.test(brown.to.center, lob.to.center)

\REM{how about sentence length?}
\end{alltt}

\end{frame}

\begin{frame}[fragile]
  \frametitle{Is word length correlated with token count?}

\begin{alltt}
\REM{average word length by tokens and types almost identical:}

> plot(brown$towl, brown$tywl)
> cor.test(brown$towl, brown$tywl)
> cor.test(brown$towl, brown$tywl,
  method="spearman")

\REM{correlation with token count}

> plot(brown$to, brown$towl)
> cor.test(brown$to, brown$towl)
\end{alltt}

\end{frame}

\end{document}
