%%
%%  some useful macro definitions & configuration settings
%%

\usepackage{latexsym}
\usepackage{graphicx}
\usepackage{url}
\usepackage{alltt}              % code examples with nicely formatted comments
\usepackage{xcolor}
\usepackage{pifont}
\usepackage{xspace}
\usepackage{array,booktabs}

\definecolor{quietred}{rgb}{.6,.2,.1}
\definecolor{quietblue}{rgb}{.1,.2,.7}
\definecolor{brightred}{rgb}{.9,0,.1}

%% > plot(x,y)      \REM{this produces a scatterplot}
\newcommand{\REM}[1]{\textsf{\small\color{quietred}\# #1}}

%% nice colour for R output: \begin{Rout} .. \end{Rout}
\newenvironment{Rout}{%
  \begin{footnotesize}\color{quietblue}\bfseries}{%
  \color{black}\mdseries\end{footnotesize}}

%% ... This is something \h{important}. ...
\newcommand<>{\h}[1]{\textbf#2{\color#2{quietred}#1}}
\newcommand<>{\hh}[1]{\textbf#2{\color#2{brightred}#1}}

%% cite \textcite{some text} in roman italic font
\newcommand{\textcite}[1]{\textrm{\textit{#1}}}

%% how can you live without the arrow (\so) and the hand (\hand) ?
\newcommand{\so}{\ding{234}\xspace}
\newcommand{\So}{\hh{\ding{229}}\xspace}
\newcommand{\hand}{\ding{43}\xspace}

%% \p{X=k};  \pC{X=k}{Y=l};  \bigp{X_i = k};   \pscale{\frac{Z}{S^2}};
%% probability P(X=k) and conditional probability P(X=k|Y=l), also with larger or scaled parentheses
%% \p[\theta]{X=k};  \pC[\text{interpolated}]{X=k}{Y=l};  ...
%% with optional subscripts (for model probability, null probability, etc.)
\newcommand{\p}[2][]{\mathop{\text{Pr}_{#1}}(#2)}
\newcommand{\pscale}[2][]{\mathop{\text{Pr}_{#1}}\!\left(#2\right)}
\newcommand{\bigp}[2][]{\mathop{\text{Pr}_{#1}}\bigl(#2\bigr)}
\newcommand{\pC}[3][]{\p[#1]{#2\,|\,#3}} 
\newcommand{\pCscale}[3][]{\pscale[#1]{#2\left|\,#3\right.\!}} 
\newcommand{\bigpC}[3][]{\bigp[#1]{#2\bigm|#3}} 

%% \Exp{X};  \Var{X};  \Exp[0]{X};  \Var[0]{X};  
%% \bigExp{X}; \bigVar{X}; \Expscale{X};  \Varscale{X};
%% expectation E[X] and variance V[X], expectation and variance under null hypothesis, 
%% and variants with largeer or scaled brackets
\newcommand{\Exp}[2][]{\text{E}_{#1}[#2]}
\newcommand{\Var}[2][]{\mathop{\text{Var}}_{#1}[#2]}
\newcommand{\bigExp}[2][]{\text{E}_{#1}\!\bigl[#2\bigr]}
\newcommand{\bigVar}[2][]{\mathop{\text{Var}}_{#1}\bigl[#2\bigr]}
\newcommand{\Expscale}[2][]{\text{E}_{#1}\left[#2\right]}
\newcommand{\Varscale}[2][]{\mathop{\text{Var}}_{#1}\left[#2\right]}
