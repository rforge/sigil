%% \documentclass[handout,t]{beamer} % HANDOUT
%% \documentclass[handout,notes=show,t]{beamer} % NOTES
\documentclass[t]{beamer} % SLIDES

\usetheme{SIGIL}
\usepackage{beamer-tools-sigil}

\input{lib/math}  % basic mathematical notation
\input{lib/stat}  % notation for probability theory and statistics
\input{lib/vector}% convenience macros for vectors and matrices

%%%
%%% local configuration adjustments
%%%

%%% You can change pre-defined colours here, override built-in macros from the
%%% style definition and standard library, as well as define macros needed by
%%% all local documents.

%%% e.g. adjust counterpoint (dark green) for data projectors where greens are
%%% far too bright, as well as green component of light colour and pure green
%%% (of course, it's a better solution to adjust the gamma settings of your monitor)
%%
%% \definecolor{counterpoint}{rgb}{.1, .3, 0}
%% \definecolor{light}{rgb}{.45, .3, .55}
%% \definecolor{puregreen}{rgb}{0, .35, 0}

%% ----- extra packages we need to load

\usepackage{tikz}
\usepackage{alltt}              % code examples with nicely formatted comments
\usepackage{rotating}


%% ----- author and copyright messages (so updates are automatically inserted into all files)
\newcommand{\sigilauthors}{%
  \author[SIGIL]{Designed by Stefan Evert\inst{1} and Marco Baroni\inst{2}}
  \institute[Evert \& Baroni]{
    \inst{1}Computational Corpus Linguistics Group\\
    Friedrich-Alexander-Universit�t Erlangen-N�rnberg, Germany
    \and
    \inst{2}Center for Mind/Brain Sciences (CIMeC)\\
    University of Trento, Italy
  }
}
\newcommand{\sigilcopyright}{%
  \date[sigil.r-forge.r-project.org]{%
    \primary{\footnotesize\url{http://SIGIL.r-forge.r-project.org/}}\\
    \light{\tiny Copyright \textcopyright\ 2007--2018 Evert \& Baroni}}
}

%% ----- automatically show TOC reminder at beginning of each subsection
\AtBeginSubsection[]
{
  \begin{frame}
    \frametitle{Outline}
    \tableofcontents[current,currentsubsection]
  \end{frame}
}

%% ----- some useful macros for the SIGIL course

\newenvironment{Rcode}[1][]{%
  \setbeamercolor{block title}{fg=counterpoint,bg=counterpoint!15!white}%
  \setbeamercolor{block body}{bg=counterpoint!5!white}\small%
  \begin{block}{#1}\begin{alltt}\ungap[1]}{%
      \ungap[1]\end{alltt}\end{block}} % \end{alltt} ... to deconfuse emacs

%% use \sbox{\Rbox} ... \usebox{\Rbox} to insert arbitray latex into Rcode environment
\newsavebox{\Rbox}

%% > plot(x,y)      \REM{this produces a scatterplot}
\newcommand{\REM}[2][\small]{\textsf{#1\color{primary}\# #2}}

%% nice colour for R output: \begin{Rout} .. \end{Rout}
\newenvironment{Rout}[1][\footnotesize]{%
  \begin{footnotesize}#1\color{secondary}\bfseries}{%
    \color{black}\mdseries\end{footnotesize}}

%% rotated column labels for table (to fit long text into narrow columns
\newcommand{\rotLabel}[2][60]{\begin{rotate}{#1}#2\end{rotate}}
 % local adjustments to configuration and macros

%%%%%%%%%%%%%%%%%%%%%%%%%%%%%%%%%%%%%%%%%%%%%%%%%%%%%%%%%%%%%%%%%%%%%%
%% Titlepage

\title[7.\ Multivariate Analysis]{Unit 7: Multivariate Analysis}
\subtitle{Statistics for Linguists with R -- A SIGIL Course}
\sigilauthors
\date[sigil.r-forge.r-project.org]{%
  \light{\tiny \sigilcopyright}}

\begin{document}

\frame{\titlepage}

%%%%%%%%%%%%%%%%%%%%%%%%%%%%%%%%%%%%%%%%%%%%%%%%%%%%%%%%%%%%%%%%%%%%%%

\section*{Outline}
\frame{ 
  \frametitle{Outline}
  \tableofcontents
}

%%%%%%%%%%%%%%%%%%%%%%%%%%%%%%%%%%%%%%%%%%%%%%%%%%%%%%%%%%%%%%%%%%%%%%
\section{Introduction}

%%%%%%%%%%%%%%%%%%%%%%%%%%%%%%%%%%%%%%%%%%
\subsection{Multivariate analysis}

\begin{frame}
  \frametitle{What is multivariate analysis?}

  \begin{itemize}
  \item Univariate statistics
    \begin{itemize}
    \item focus on a single variable of interest (at a time)
    \item estimate population parameters ($\pi$, $\mu$, $\sigma^2$, \ldots)
    \item comparison of two or more groups
    \end{itemize}
  \item<2-> Bivariate statistics
    \begin{itemize}
    \item focus on interdependencies of two variables
    \item correlation \& co-occurrence
    \end{itemize}
  \item<3-> Regression modelling
    \begin{itemize}
    \item predict single target variable (``dependent'')
    \item based on multiple other variables (``independent'')
    \end{itemize}
  \item<4-> Multivariate statistics
    \begin{itemize}
    \item combined effects of many variables
    \item correlations \& distribution patterns
    \item often ``unsupervised'': no target variable or comparison groups
    \end{itemize}
  \end{itemize}
\end{frame}

\begin{frame}
  \frametitle{Application examples}

  \begin{itemize}
  \item Register variation \citep{Biber:88,Biber:93}
  \item Translation studies\\ \citep{Evert:Neumann:16,DeSutter:Delaere:Plevoets:12}
  \item Stylometry: authorshop attribution \citep{Evert:etc:17}
  \item Dialectology \citep{Speelman:Grondelaers:Geeraerts:03}
  \item Historical linguistics \citep{Sagi:Kaufmann:Clark:09,Perek:18}
  \item Identification of confounding variables \citep{Tummers:Speelman:Geeraerts:14}
  \item Linguistic productivity \citep{Jenset:McGillivray:12}
  \item Correspondence analysis \citep{Greenacre:07}
  \item Distributional semantics (see \href{http://wordspace.collocations.de/doku.php/course:esslli2018:start}{\secondary{ESSLLI course}})
  \end{itemize}
\end{frame}

%%%%%%%%%%%%%%%%%%%%%%%%%%%%%%%%%%%%%%%%%%
\subsection{Setting up}

\begin{frame}
  \frametitle{R packages}
  % \framesubtitle{}

  Required R packages:
  \begin{itemize}
  \item \code{corpora} ($\geq$ 0.5)
  \item \code{wordspace} ($\geq$ 0.2)
  \end{itemize}

  Recommended packages:
  \begin{itemize}
  \item \code{ggplot2}, \code{reshape2} \ldots\ for plotting feature weights
  \item \code{rgl} \ldots\ for interactive 3-d visualization
  \item \code{Hotelling}, \code{ellipse} \ldots\ for significance testing
  \item \code{e1071} \ldots\ for machine learning (SVM)
  \item \code{Rtsne} \ldots\ for low-dimensional maps
  \item \code{ca} \ldots\ for correspondence analysis
  % \item \code{} \ldots\ for
  \item[]
  \end{itemize}

  \hand{} install with package manager in RStudio or R GUI
\end{frame}

\begin{frame}[fragile]
  \frametitle{Code \& data sets}
  % \framesubtitle{}

  Download additional code \& data sets from SIGIL homepage:
  \begin{itemize}
  \item \href{http://www.stefan-evert.de/SIGIL/sigil_R/materials/multivar_utils.R}{\code{multivar\_utils.R}}
  \item \href{http://www.stefan-evert.de/SIGIL/sigil_R/data/unit7_data.rda}{\code{unit7\_data.rda}}
  \item[]
  \end{itemize}

  \hand{} put all files in RStudio project directory (or working directory)

  \gap[1]
\begin{Rcode}
> library(corpora)           \REM{basic utilities and some data sets}
> library(wordspace)         \REM{for large and sparse matrices}

> source("multivar_utils.R") \REM{additional functions}

> load("unit7_data.rda", verbose=TRUE) \REM{further data sets}
\end{Rcode}
\end{frame}

\begin{frame}
  \frametitle{Overview of data sets}

  \begin{itemize}
  \item 65 Biber features for British National Corpus
    \begin{itemize}
    \item \code{BNCbiber} = $4048\times 65$ feature matrix
    \item \code{BNCmeta} = complete metadata table
    \item extensive documentation with \code{?BNCbiber}, \code{?BNCmeta}
    \item[]
    \end{itemize}
  \item 67 Biber features for Brown Family corpora
    \begin{itemize}
    \item \code{BrownBiber\_Matrix} = $3500 x 67$ feature matrix
    \item \code{BrownBiber\_Meta} = metadata table
    \item features are Biber-scaled z-scores obtained with MAT v1.3\\
      {\footnotesize\secondary{\url{http://sites.google.com/site/multidimensionaltagger/}}}
    \item see tagger manual for feature definitions
    \end{itemize}
  \end{itemize}
\end{frame}

\begin{frame}
  \frametitle{Overview of data sets}

  \begin{itemize}
  \item 27 SFL-inspired features for translation pairs (CroCo corpus)
    \begin{itemize}
    \item \code{CroCo\_Matrix} = $452\times 27$ feature matrix
    \item \code{CroCo\_Meta} = metadata table
    \item \code{CroCo\_orig2trans} = row numbers of translation pairs
    \item data from \citet{Evert:Neumann:16}
    \item[]
    \end{itemize}
  \item Literary authorship attribution with $\Delta$ measures
    \begin{itemize}
    \item data: sparse document-term matrices for 20,000 most frequent words (mfw) as \texttt{wordspace} DSM objects
    \item \code{Delta\$DE} = $75\times 20000$ matrix (German novels, 25 authors)
    \item \code{Delta\$EN} = $75\times 20000$ matrix (English novels, 25 authors)
    \item \code{Delta\$FR} = $75\times 20000$ matrix (French novels, 25 authors)
    \item \code{Delta\$DE\$rows}, \code{Delta\$EN\$rows}, \ldots\ = metadata tables
    \item \code{DeltaLemma} = lemmatized version
    \item data from \citet{Jannidis:etc:15,Evert:etc:17}
    \end{itemize}
  \end{itemize}  
\end{frame}

\begin{frame}
  \frametitle{Overview of data sets}

  \begin{itemize}
  \item 19 type-token complexity measures for $\Delta$ corpus
    \begin{itemize}
    \item complexity scores for 10,000-token text slices from 75 novels 
    \item \code{DeltaComplexity\$DE\$Matrix} = $996\times 19$ matrix (German)
    \item \code{DeltaComplexity\$EN\$Matrix} = $1147\times 19$ matrix (English)
    \item \code{DeltaComplexity\$FR\$Matrix} = $679\times 19$ matrix (French)
    \item \code{DeltaComplexity\$DE\$Meta}, \ldots\ = metadata tables
    \item can be used to study correlational patterns between measures
    \item[]
    \end{itemize}
  \item 7 syntactic complexity measures for 969 German novels
    \begin{itemize}
    \item \code{SyntacticComplexity\_Matrix} = $969\times 7$ feature matrix
    \item \code{SyntacticComplexity\_Meta} = metadata tables
    \item can be used to compare high-brow against low-brow literature
    \end{itemize}
  \end{itemize}  
\end{frame}


%%%%%%%%%%%%%%%%%%%%%%%%%%%%%%%%%%%%%%%%%%%%%%%%%%%%%%%%%%%%%%%%%%%%%%
\section{Mathematical background}

%%%%%%%%%%%%%%%%%%%%%%%%%%%%%%%%%%%%%%%%%%
\subsection{Feature matrix}

\begin{frame}[fragile]
  \frametitle{Feature matrix}
  % \framesubtitle{}

  \h{Feature matrix} records quantitative features for each text 

  \gap[1]
  \begin{center}
  \(
  \mathbf{M} = 
  \begin{bmatrix}
    \cdots & \vm_1 & \cdots \\
    \cdots & \vm_2 & \cdots \\
    & \vdots & \\
    & \vdots & \\
    \cdots & \vm_k & \cdots \\
  \end{bmatrix}
  \)
  \hspace{3mm}
  \begin{footnotesize}
    \setlength{\arrayrulewidth}{0.5pt}
    \begin{tabular}[c]{r|c@{$\;$}*{4}{|@{$\;$}c@{$\;$}}|}
      \multicolumn{1}{c}{}
      & \multicolumn{1}{c}{\rotLabel[30]{nominal}}
      & \multicolumn{1}{c}{\rotLabel[30]{pass}}
      & \multicolumn{1}{c}{\rotLabel[30]{prep}}
      & \multicolumn{1}{c}{\rotLabel[30]{subord}}
      & \multicolumn{1}{c}{\rotLabel[30]{ttr}} \\
      \cline{2-6}
      orig$_1$   & 1.205 & 5.013 & 6.883  & 4.483 & 1.285 \\ 
      \cline{2-6}
      orig$_2$   & 0.738 & 2.537 & 6.486  & 6.157 & 1.714 \\ 
      \cline{2-6}
      orig$_3$   & 1.252 & 4.462 & 8.463  & 4.785 & 2.476 \\ 
      \cline{2-6}
      orig$_4$   & 1.105 & 2.899 & 8.119  & 3.966 & 1.519 \\ 
      \cline{2-6}
      orig$_5$   & 1.764 & 4.268 & 7.167  & 3.947 & 1.792 \\ 
      \cline{2-6}
      orig$_8$   & 1.545 & 7.268 & 7.461  & 5.455 & 1.572 \\ 
      \cline{2-6}
      trans$_1$  & 0.463 & 2.208 & 6.297  & 6.089 & 2.339 \\ 
      \cline{2-6}
      trans$_2$  & 1.131 & 2.597 & 6.307  & 4.844 & 1.810 \\ 
      \cline{2-6}
      trans$_4$  & 0.935 & 1.744 & 7.098  & 4.012 & 1.403 \\ 
      \cline{2-6}
      trans$_5$  & 0.867 & 3.604 & 7.511  & 5.154 & 1.902 \\ 
      \cline{2-6}
      trans$_7$  & 1.387 & 4.290 & 8.211  & 3.998 & 1.822 \\ 
      \cline{2-6}
    \end{tabular}
  \end{footnotesize}
  \end{center}

\begin{Rcode}
> M <- MultiVar_Matrix
> M
\end{Rcode}
\end{frame}

%%%%%%%%%%%%%%%%%%%%%%%%%%%%%%%%%%%%%%%%%%
\subsection{Distance metric}

\begin{frame}
  \frametitle{Geometric distance = metric}
  %% \framesubtitle{}

  \begin{columns}[T]
    \begin{column}{60mm}
      \begin{itemize}
      \item \hh{Distance} between vectors $\vu, \vv \in \setR^n$ \so
        (dis)similarity
        \begin{itemize}
        \item $\vu = (u_1, \ldots, u_n)$
        \item $\vv = (v_1, \ldots, v_n)$
        \end{itemize}
      \item<2-> \h{Euclidean} distance $\dist[2]{\vu}{\vv}$
      \item<3-> ``City block'' \h{Manhattan} distance $\dist[1]{\vu}{\vv}$
      \item<4-> Both are special cases of the \h{Minkowski} $p$-distance
        $\dist[p]{\vu}{\vv}$ (for $p\in [1, \infty]$)
      \end{itemize}
    \end{column}
    \begin{column}{45mm}
      \includegraphics[width=45mm]{img/7_distance_examples}
    \end{column}
  \end{columns}
  \gap[.5]
  \only<beamer:2| handout:0>{%
    \[ \dist[2]{\vu}{\vv} \coloneq \sqrt{(u_1 - v_1)^2 + \dots + (u_n - v_n)^2} \] }
  \only<beamer:3| handout:0>{%
    \[ \dist[1]{\vu}{\vv} \coloneq \abs{u_1 - v_1} + \dots + \abs{u_n - v_n} \] }
  \only<beamer:4-| handout:1>{%
    \[ \dist[p]{\vu}{\vv} \coloneq \bigl(
    \abs{u_1 - v_1}^p + \dots + \abs{u_n - v_n}^p
    \bigr)^{1/p} \] }
  \only<beamer:5-| handout:1>{%
    \[ \dist[\infty]{\vu}{\vv} = \max \bigset{\abs{u_1 - v_1}, \ldots, \abs{u_n - v_n}} \] }
\end{frame}

\begin{frame}
  \frametitle{Geometric distance = metric}
  %% \framesubtitle{}

  \begin{columns}[T]
    \begin{column}{60mm}
      \begin{itemize}
      \item \hh{Distance} between vectors $\vu, \vv \in \setR^n$ \so
        (dis)similarity
        \begin{itemize}
        \item $\vu = (u_1, \ldots, u_n)$
        \item $\vv = (v_1, \ldots, v_n)$
        \end{itemize}
      \item \h{Euclidean} distance $\dist[2]{\vu}{\vv}$
      \item ``City block'' \h{Manhattan} distance $\dist[1]{\vu}{\vv}$
      \item Extension of $p$-distance  $\dist[p]{\vu}{\vv}$ (for $0\leq p\leq 1$)
      \end{itemize}
    \end{column}
    \begin{column}{45mm}
      \includegraphics[width=45mm]{img/7_distance_examples}
    \end{column}
  \end{columns}
  \gap[.5]
  \[ \dist[p]{\vu}{\vv} \coloneq \abs{u_1 - v_1}^p + \dots + \abs{u_n - v_n}^p \]
  \[ \dist[0]{\vu}{\vv} = \# \bigsetdef{i}{u_i \neq v_i} \]
\end{frame}

\begin{frame}
  \frametitle{Distance and vector length = norm}
  %% \framesubtitle{}

  \begin{columns}[T]
    \begin{column}{50mm}
      \begin{itemize}
      \item<1-> Intuitively, distance $\dist{\vu}{\vv}$ should correspond to
        length $\norm{\vu-\vv}$ of displacement vector $\vu - \vv$
        \begin{itemize}
        \item $\dist{\vu}{\vv}$ is a \h{metric}
        \item $\norm{\vu-\vv}$ is a \h{norm}
        \item $\norm{\vu} = \bigdist{\vu}{\vnull}$
        \end{itemize}
      \item<2-> Any norm-induced metric is \h{translation-invariant}%
        \gap
      \item<3-> $\dist[p]{\vu}{\vv} = \norm[p]{\vu-\vv}$
      \end{itemize}
    \end{column}
    \begin{column}{50mm}
      \includegraphics[width=50mm]{img/7_distance_norm}
    \end{column}
  \end{columns}

  \begin{itemize}
  \item<3-> \h{Minkowski $p$-norm} for $p\in [1,\infty]$ (not $p < 1$):
    \[
    \norm[p]{\vu} \coloneq \bigl(\abs{u_1}^p + \dots + \abs{u_n}^p\bigr)^{1/p}
    \]
  \end{itemize}
\end{frame}

\begin{frame}[fragile]
  \frametitle{Computing distances}

Compute distances between all pairs of texts:
\begin{Rcode}
> round(dist(M), 2)  \REM{returns a triangular `dist' object}
> round(dist(M, method="manhattan"), 2) \REM{Manhattan metric}
\end{Rcode}

\gap[1]\pause
Use \texttt{wordspace} function for additional metrics:
\begin{Rcode}
> dist.matrix(M, method="mink", p=0.5)  \REM{full matrix}
> dist.matrix(M, method="mink", p=0.5, as.dist=TRUE)
\end{Rcode}

\gap[1]\pause
Standardize features for equal contribution to Euclidean metric:
\begin{Rcode}
> Z <- scale(M)     \REM{matrix of z-scores}
> round(dist(Z), 2) \REM{default: Euclidean metric}
\end{Rcode}
\end{frame}

%%%%%%%%%%%%%%%%%%%%%%%%%%%%%%%%%%%%%%%%%%
\subsection{Orthogonal projection}

\begin{frame}
  \frametitle{Linear subspace \& basis}
  
  \begin{itemize}
  \item A linear \h{subspace} $B\subseteq \setR^n$ of rank $r\leq n$ is spanned by a set of $r$ linearly independent basis vectors
    \[
    B = \set{\vb_1, \ldots, \vb_r}
    \]
  \item<2-> Every point $\vu$ in the subspace is a unique linear combination of the basis vectors
    \[
    \vu = x_1 \vb_1 + \ldots + x_r \vb_r
    \]
    with coordinate vector $\vx\in \setR^r$
  \item<3-> Basis matrix $\mathbf{V} \in \setR^{n\times r}$ with column vectors $\vb_i$:
    \[
    \vu = \mathbf{V} \vx 
    \]
  \end{itemize}
\end{frame}

\begin{frame}
  \frametitle{Linear subspace \& basis}
  
  \begin{itemize}
  \item Basis matrix $\mathbf{V} \in \setR^{n\times r}$ with column vectors $\vb_i$:
    \[
    \vu = x_1 \vb_1 + \ldots + x_r \vb_r = \mathbf{V} \vx
    \]
  \end{itemize}

  \[
  \begin{array}{ccccc}
    \begin{bmatrix}
      x_1 b_{11} + \ldots + x_r b_{1r} \\
      x_1 b_{21} + \ldots + x_r b_{2r} \\
      \vdots\\
      x_1 b_{n1} + \ldots + x_r b_{nr}
    \end{bmatrix}
    & = &
    \begin{bmatrix}
      b_{11} & \cdots & b_{1r} \\
      b_{21} & \cdots & b_{2r} \\
      \vdots & & \vdots \\
      b_{n1} & \cdots & b_{nr}
    \end{bmatrix}
    & \cdot &
    \begin{bmatrix}
      x_1 \\
      \vdots \\
      x_r
    \end{bmatrix} \\
    \\
    \mathbf{u} & = & \mathbf{V} & \cdot & \mathbf{x} \\
    (n\times 1) & & (n\times \primary{r}) & & (\primary{r}\times 1)
  \end{array}
  \]

\end{frame}

\begin{frame}
  \frametitle{Orthonormal basis}

  \begin{itemize}
  \item Particularly convenient with orthonormal basis:
    \begin{align*}
      \norm[2]{\vb_i} &= 1 \\
      \vb_i^T \vb_j &= 0 && \text{for } i\neq j
    \end{align*}
  \item Corresponding basis matrix $\mathbf{V}$ is (column)-\hh{orthogonal}
    \[
      \mathbf{V}^T \mathbf{V} = \mathbf{I}_r
    \]
    and defines a \primary{Cartesian coordinate system} in the subspace
  \end{itemize}
\end{frame}

\begin{frame}
  \frametitle{An aside: Matrix multiplication}
  %% \framesubtitle{}

  \ungap
  \[
  \begin{array}{ccccc}
    \begin{bmatrix}
      & \only<beamer:2| handout:0>{a_{ij}} & \only<beamer:1| handout:0>{a_{ij}} & \\
      \only<beamer:3| handout:1>{a_{ij}} & & & \\
      & & &
    \end{bmatrix}
    & = &
    \begin{bmatrix}
      \only<beamer:1-2| handout:0>{b_{i1}} & \only<beamer:1-2| handout:0>{\cdots} & \only<beamer:1-2| handout:0>{b_{in}} \\
      \only<beamer:3| handout:1>{b_{i1}} & \only<beamer:3| handout:1>{\cdots} & \only<beamer:3| handout:1>{b_{in}} \\
      & &  
    \end{bmatrix}
    & \cdot &
    \begin{bmatrix}
     \only<beamer:3| handout:1>{c_{1j}}   & \only<beamer:2| handout:0>{c_{1j}} & \only<beamer:1| handout:0>{c_{1j}} &\\
     \only<beamer:3| handout:1>{\vdots}   & \only<beamer:2| handout:0>{\vdots} & \only<beamer:1| handout:0>{\vdots} & \\
     \only<beamer:3| handout:1>{\vdots}   & \only<beamer:2| handout:0>{\vdots} & \only<beamer:1| handout:0>{\vdots} & \\
     \only<beamer:3| handout:1>{c_{nj}}   & \only<beamer:2| handout:0>{c_{nj}} & \only<beamer:1| handout:0>{c_{nj}} & 
    \end{bmatrix} \\
    \\
    \mathbf{A} & = & \mathbf{B} & \cdot & \mathbf{C} \\
    (k\times m) & & (k\times \primary{n}) & & (\primary{n}\times m)
  \end{array}
  \]
  \begin{itemize}
  \item $\mathbf{B}$ and $\mathbf{C}$ must be \h{conformable} (in dimension \primary{$n$})
  \item Element $a_{ij}$ is the inner product of the $i$-th row of $\mathbf{B}$ and the $j$-th column of $\mathbf{C}$
    \[
    a_{ij} = b_{i\primary{1}} c_{\primary{1}j} + \ldots + b_{i\primary{n}} c_{\primary{n}j} = \sum_{t=1}^n b_{i\primary{t}} c_{\primary{t}j}
    \]
  \end{itemize}
\end{frame}


\begin{frame}
  \frametitle{The mathematics of projections}
  %% \framesubtitle{}

  \begin{columns}[c]
    \begin{column}{50mm}
      \begin{itemize}
      \item 1-d subspace spanned by basis vector
        $\norm[2]{\vb} = 1$
      \item For any point $\vu$, we have
        \[
          \cos \varphi
          = \frac{\vb^T \vu}{\norm[2]{\vb}\cdot \norm[2]{\vu}}
          = \frac{\vb^T \vu}{\norm[2]{\vu}}          
        \]
      \item<2-> Trigonometry: coordinate of point on the line is
        $x = \norm[2]{\vu}\cdot \cos\varphi = \vb^T \vu$
      \end{itemize}
    \end{column}
    \begin{column}{50mm}
      \includegraphics[width=50mm]{img/7_cosine_projection}
    \end{column}
  \end{columns}
  
  \begin{itemize}
  \item<3-> The projected point in original space is then given by
    \[
      \vb\cdot x = \vb (\vb^T \vu) = (\vb \vb^T) \vu = \mathbf{P} \vu
    \]
    where $\mathbf{P}$ is a \h{projection matrix} of rank 1
  \end{itemize}
\end{frame}

\begin{frame}
  \frametitle{The mathematics of projections}
  %% \framesubtitle{}

  \begin{itemize}
  \item For an orthogonal basis matrix $\mathbf{V}$ with columns
    $\vb_1, \ldots, \vb_r$, the projection into the rank-$r$ subspace $B$ is
    given by
    \[
      \mathbf{P}\vu = \left(\sum_{i=1}^r \vb_i \vb_i^T\right) \vu
      = \mathbf{V} \mathbf{V}^T \vu
    \]
    and its subspace coordinates are $\vx = \mathbf{V}^T \mathbf{u}$
  \item<2-> Projection can be seen as decomposition into the projected vector
    and its orthogonal complement
    \[
      \vu = \mathbf{P} \vu + (\vu - \mathbf{P} \vu)
      = \mathbf{P} \vu + (\mathbf{I} - \mathbf{P}) \vu
      = \mathbf{P} \vu + \mathbf{Q} \vu
    \]
  \item<3-> Because of orthogonality, this also applies to the squared
    Euclidean norm (according to the Pythagorean theorem)
    \[
      \norm{\vu}^2 = \norm{\mathbf{P} \vu}^2 + \norm{\mathbf{Q} \vu}^2
    \]
  \end{itemize}
\end{frame}

\begin{frame}
  \frametitle{Optimal projections and subspaces}
  %% \framesubtitle{}

  \begin{itemize}
  \item Orthogonal decomposition of squared distances btw vectors
    \[
      \norm{\vu - \vv}^2 = \norm{\mathbf{P} \vu - \mathbf{P} \vv}^2 + \norm{\mathbf{Q} \vu - \mathbf{Q} \vv}^2
    \]
  \end{itemize}

  \begin{columns}[c]
    \begin{column}{60mm}
      \begin{itemize}
      \item<2-> Define projection \hh{loss} as difference btw squared distances
        \begin{align*}
          & \bigabs{\,\norm{\mathbf{P}(\vu - \vv)}^2 - \norm{\vu - \vv}^2\,} \\
          =\;& \norm{\vu - \vv}^2 - \norm{\mathbf{P}(\vu - \vv)}^2 \\
          =\;& \norm{\mathbf{Q}(\vu - \vv)}^2
        \end{align*}
      \item<3-> Projection quality measure:
        \[
          R^2 = \frac{\norm{\mathbf{P}(\vu - \vv)}^2}{\norm{\vu - \vv}^2}
        \]
      \end{itemize}
    \end{column}
    \begin{column}{40mm}
      \hspace*{-5mm}%
      \includegraphics[width=45mm]{img/7_projection_loss}      
    \end{column}
  \end{columns}
\end{frame}


% %%%%%%%%%%%%%%%%%%%%%%%%%%%%%%%%%%%%%%%%%%%%%%%%%%%%%%%%%%%%%%%%%%%%%%
% \section{}

% %%%%%%%%%%%%%%%%%%%%%%%%%%%%%%%%%%%%%%%%%%
% \subsection{}

% \begin{frame}
%   \frametitle{}
%   % \framesubtitle{}

% \end{frame}

% \begin{frame}[fragile]
%   \frametitle{}
%   %% \framesubtitle{}

%   % \ungap[1]
%   \begin{alltt}
%   \end{alltt}
% \end{frame}

%%%%%%%%%%%%%%%%%%%%%%%%%%%%%%%%%%%%%%%%%%%%%%%%%%%%%%%%%%%%%%%%%%%%%%
%% References (if any)

\frame[allowframebreaks]{
  \frametitle{References}
  \bibliographystyle{natbib-stefan}
  \begin{scriptsize}
    \bibliography{sigil}
  \end{scriptsize}
}

\end{document}
