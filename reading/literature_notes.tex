\documentclass[a4paper]{article}

\usepackage{vmargin}
\setpapersize[portrait]{A4}
\setmarginsrb{30mm}{10mm}{30mm}{20mm}% left, top, right, bottom
{12pt}{15mm}% head heigth / separation
{0pt}{15mm}% bottom height / separation
%% \setmargnohfrb{30mm}{20mm}{20mm}{20mm}

\setlength{\parindent}{0mm}
\setlength{\parskip}{\medskipamount}

\usepackage[english]{babel}
\usepackage[utf8]{inputenc}

\usepackage[T1]{fontenc}
%% \usepackage{textcomp}  % this can break some outline symbols in CM fonts, use only if absolutely necessary

%% \usepackage{lmodern}   % type1 computer modern fonts in T1 encoding
% \usepackage{stefan-fonts}  % commercial Charter fonts with full math support
\usepackage{chmath}

%% \usepackage{mathptmx}  % use Adobe Times as standard font with simulated math support
%% \usepackage[sc,osf]{mathpazo}  % use Adobe Palatino as standard font with simulated math support

%% \usepackage{pifont}
%% \usepackage{eucal}

\usepackage{amsmath,amssymb,amsthm}
\usepackage{graphicx,rotating}
\usepackage{array,hhline,booktabs}
\usepackage{xspace}
\usepackage{url}
%% \usepackage{ifthen,calc,hyphenat}

\usepackage{xcolor,tikz}
\usepackage[textwidth=25mm,textsize=footnotesize,colorinlistoftodos,backgroundcolor=orange!80]{todonotes} % [disable] to hide all TODOs

\usepackage{natbib}
\bibpunct{(}{)}{;}{a}{}{,}

%%
%% some useful macros: mathematical notation etc.
%%

%% abbreviations for logic symbols
\renewcommand{\implies}{\Rightarrow}
\newcommand{\equivalent}{\Leftrightarrow}

%% abbreviations for common number spaces
\newcommand{\setN}[1][]{\mathbb{N}_{#1}} % allows \setN and \setN[0]
\newcommand{\setZ}{\mathbb{Z}}
\newcommand{\setQ}{\mathbb{Q}}
\newcommand{\setR}{\mathbb{R}}

%% sets and (sub-)sets defined by condition
\newcommand{\set}[1]{\{#1\}}
\newcommand{\setdef}[2]{\set{#1\,|\,#2}}
\newcommand{\bigset}[1]{\bigl\{#1\bigr\}}
\newcommand{\bigsetdef}[2]{\bigset{#1\bigm|#2}}
\newcommand{\setscale}[1]{\left\{#1\right\}}
\newcommand{\setdefscale}[2]{\setscale{#1\left|\,#2\right.}}

%% absolute value and norm
\newcommand{\abs}[1]{\lvert #1\rvert}
\newcommand{\bigabs}[1]{\bigl\lvert #1\bigr\rvert}
\newcommand{\absscale}[1]{\left\lvert #1\right\rvert}
\newcommand{\norm}[2][]{\lVert #2\rVert_{#1}}
\newcommand{\bignorm}[2][]{\bigl\lVert #2\bigr\rVert_{#1}}
\newcommand{\normscale}[2][]{\left\lVert #2\right\rVert_{#1}}

%% complement set (with optional index)
\newcommand{\compl}[1][]{\mathcal{C}^{#1}}

%% power set: \powerset{\Sigma^*}
\newcommand{\powerset}[1]{\mathcal{P}(#1)}

%% uparrow: a \ua b = direct dominance in ordered tree
\newcommand{\ua}{\uparrow}

%% left-right arrow: this $\lra$ that
\newcommand{\lra}{\leftrightarrow}

%% expanded engineering notation: 4.2\x\e+5
\newcommand{\e}[2]{10^{\ifthenelse{\equal{#1}{+}}{}{#1}#2}}
\newcommand{\x}{\cdot}

%% arg max & min: \argmax_{x\in C}, \argmin_{x\in C}
\newcommand{\argmax}{\mathop{\text{arg~max}}}
\newcommand{\argmin}{\mathop{\text{arg~min}}}

%% infinitesimal elements: \dx, \dy = \dX{y}, \dz
\newcommand{\dX}[1]{\,\mathit{d{#1}}}
\newcommand{\dx}{\dX{x}}
\newcommand{\dy}{\dX{y}}
\newcommand{\dz}{\dX{z}}

%%% Local Variables: 
%%% mode: latex
%%% TeX-master: ""
%%% End: 

%%
%% some useful macros: statistical notation
%%

%% \p{X=k};  \pC{X=k}{Y=l};  \bigp{X_i = k};   \pscale{\frac{Z}{S^2}};
%% probability P(X=k) and conditional probability P(X=k|Y=l), also with larger or scaled parentheses
%% \p[\theta]{X=k};  \pC[\text{interpolated}]{X=k}{Y=l};  ...
%% with optional subscripts (for model probability, null probability, etc.)
\newcommand{\p}[2][]{\mathop{\mathrm{Pr}_{#1}}(#2)}
\newcommand{\pscale}[2][]{\mathop{\mathrm{Pr}_{#1}}\!\left(#2\right)}
\newcommand{\bigp}[2][]{\mathop{\mathrm{Pr}_{#1}}\bigl(#2\bigr)}
\newcommand{\pC}[3][]{\p[#1]{#2\,|\,#3}} 
\newcommand{\pCscale}[3][]{\pscale[#1]{#2\left|\,#3\right.\!}} 
\newcommand{\bigpC}[3][]{\bigp[#1]{#2\!\bigm|\!#3}} 

%% \Exp{X};  \Var{X};  \Exp[0]{X};  \Var[0]{X};  
%% \bigExp{X}; \bigVar{X}; \Expscale{X};  \Varscale{X};
%% expectation E[X] and variance V[X], expectation and variance under null hypothesis, 
%% and variants with largeer or scaled brackets
\newcommand{\Exp}[2][]{\mathrm{E}_{#1}[#2]}
\newcommand{\Var}[2][]{\mathop{\mathrm{Var}}_{#1}[#2]}
\newcommand{\bigExp}[2][]{\mathrm{E}_{#1}\!\bigl[#2\bigr]}
\newcommand{\bigVar}[2][]{\mathop{\mathrm{Var}}_{#1}\bigl[#2\bigr]}
\newcommand{\Expscale}[2][]{\mathrm{E}_{#1}\left[#2\right]}
\newcommand{\Varscale}[2][]{\mathop{\mathrm{Var}}_{#1}\left[#2\right]}

%% \pihat = \hat{\pi}
%% sampling estimate for population probability \pi (may need fine-tuning)
\newcommand{\pihat}{\hat{\pi}}

%% \Entropy{X}, \Entropy{p}, \KL{p}{q}, \MI{X}{Y}
%% \bigEntropy{}, \Entropyscale{}, \bigKL{}{}, \KLscale{}{}, \bigMI{}{}, \MIscale{}{}
%% entropy, KL distance, conditional entropy and mutual information (with scaled variants)
\newcommand{\Entropy}[1]{H[{#1}]}
\newcommand{\bigEntropy}[1]{H\bigl[{#1}\bigr]}
\newcommand{\Entropyscale}[1]{H\left[{#1}\right]}
\newcommand{\KL}[2]{D({#1}\|{#2})}
\newcommand{\bigKL}[2]{D\bigl({#1}\bigm\|{#2}\bigr)}
\newcommand{\KLscale}[2]{D\left({#1}\left\|{#2}\right.\right)}
\newcommand{\MI}[2]{I[{#1};{#2}]}
\newcommand{\bigMI}[2]{I\bigl[{#1};{#2}\bigr]}
\newcommand{\MIscale}[2]{I\left[{#1};{#2}\right]}

%% \corr (correlation) and \cov (covariance) as mathop's
\newcommand{\corr}{\mathop{\mathrm{corr}}}
\newcommand{\cov}{\mathop{\mathrm{cov}}
}
%%% Local Variables: 
%%% mode: latex
%%% TeX-master: ""
%%% End: 


\title{SIGIL: Statistical Inference for Corpus Linguistics\\--- literature review \& notes ---}
\author{Stefan Evert}
\date{Typeset on \today}

\begin{document}
\maketitle

\listoftodos
\tableofcontents


%%%%%%%%%%%%%%%%%%%%%%%%%%%%%%%%%%%%%%%%%%%%%%%%%%%%%%%%%%%%%%%%%%%%%%%%
%%%%%%%%%%%%%%%%%%%%%%%%%%%%%%%%%%%%%%%%%%%%%%%%%%%%%%%%%%%%%%%%%%%%%%%%
\section{Inbox \& tasks}
\label{sec:inbox}

%%%%%%%%%%%%%%%%%%%%%%%%%%%%%%%%%%%%%%%%%%%%%%%%%%%%%%%%%%%%%%%%%%%%%%%%
\subsection{Tasks for \texttt{corpora} package}
\label{sec:tasks-corpora}

\begin{itemize}
\item Change implementation of p-values and confidence intervals to use ``central'' method \citep[$p_c$, cf.][]{Fay:10a}, which is efficient and consistent.  Explain in help pages why this is a sensible choice for corpus linguistics (see Sec.~\ref{sec:Fay2010}).
\end{itemize}

%%%%%%%%%%%%%%%%%%%%%%%%%%%%%%%%%%%%%%%%%%%%%%%%%%%%%%%%%%%%%%%%%%%%%%%%
\subsection{Tasks for SIGIL slides}
\label{sec:tasks-sigil}

\begin{itemize}
\item Mention different types of two-sided p-values and confidence intervals, which can lead to inconsistencies, with example from \citet{Fay:10a}.  Recommendation for corpus linguistics: conservative central p-values $p_c$ and matching confidence intervals, implemented in \texttt{corpora} package.  Note that most standard implementations (e.g.\ \texttt{fisher.test} and \texttt{binom.test} in R) calculate minlike p-values $p_m$ that will often differ from $p_c$.
\end{itemize}

%%%%%%%%%%%%%%%%%%%%%%%%%%%%%%%%%%%%%%%%%%%%%%%%%%%%%%%%%%%%%%%%%%%%%%%%
\subsection{Inbox}
\label{sec:inbox}

%%%%%%%%%%%%%%%%%%%%%%%%%%%%%%%%%%%%%%%%%%%%%%%%%%%%%%%%%%%%%%%%%%%%%%%%
\subsubsection{\citet{Fay:10,Fay:10a}: Exact confidence intervals for two-sided tests}
\label{sec:Fay2010}

\begin{itemize}
\item Notation: test statistic $T$ for parameter $\theta$ with observed value $t$; $f_{\theta}(t) = \pC{T=t}{\theta}$ is the likelihood; $F_{\theta}(t) = \pC{T\leq t}{\theta}$ and $\bar{F}_{\theta}(t) = \pC{T\geq t}{\theta}$ are the lower/upper tail probabilities
\item My notaton: $G_{\theta}(t) = \min\set{F_{\theta}(t), \bar{F}_{\theta}(t)}$ for the ``appropriate'' tail probability
\item Exact two-sided p-values for discrete data can be defined in different ways. \citet{Fay:10a} lists three important and widely-used methods:
  \begin{enumerate}
  \item \textbf{central} p-value: $p_c = 2\cdot G_{\theta}(t)$, clamped to $[0,1]$ (aka.\ TST = twice the smaller tail); usually easy to compute and well-behaved (monotonic in $\theta$, cf.\ Fig.~1 on p.~56); exact confidence intervals are often based on $p_c$
  \item \textbf{minlike} p-value: $p_m = \sum_{f(s)\leq f(t)} f_{\theta}(s)$; implementations of exact tests often report $p_m$ as two-sided p-value; not always well-behaved and confidence sets may have ``holes''
  \item \textbf{blaker} p-value: $p_b$ is the appropriate tail probability $G_{\theta}(t)$ plus the largest opposite tail that does not exceed $G_{\theta}(t)$ (aka.\ CT = combined tails method); \citet{Blaker:00}\todo{get copy of \citet{Blaker:00} and update details in bibtex database} presents a comprehensive analysis of this method and derives improved confidence intervals
  \end{enumerate}
\item \citet{Fay:10a} gives an excellent concise overview of two-sided p-values and confidence intervals for $2\times 2$ tables using the three methods above, with examples of inconsistencies arising if different methods are mixed (typically central confidence interval with minlike p-value)
\item All three methods and Matching confidence intervals developed by \citet{Fay:10}\todo{read \citet{Fay:10} and add notes} are implemented in the R packages \textbf{exact2x2} and \textbf{exactci}; in some cases, inconsistencies cannot be entirely avoided because ``true'' confidence sets aren't connected intervals (Fig.~3, p.~57)
\item Conclusion: central p-value $p_c$ is easy to compute, well-behaved (Fig.~1) and consistent confidence intervals can be determined efficiently; in most cases (though not always) it is more conservative than the other methods (i.e.\ $p_c > p_m, p_b$).  Therefore it makes sense to \textbf{use central p-values} and the matching confidence intervals in \textbf{corpus linguistics}
\end{itemize}

%%%%%%%%%%%%%%%%%%%%%%%%%%%%%%%%%%%%%%%%%%%%%%%%%%%%%%%%%%%%%%%%%%%%%%%%
% \subsection{}



%%%%%%%%%%%%%%%%%%%%%%%%%%%%%%%%%%%%%%%%%%%%%%%%%%%%%%%%%%%%%%%%%%%%%%%%  
%%%%%%%%%%%%%%%%%%%%%%%%%%%%%%%%%%%%%%%%%%%%%%%%%%%%%%%%%%%%%%%%%%%%%%%%
% \section{}

%%%%%%%%%%%%%%%%%%%%%%%%%%%%%%%%%%%%%%%%%%%%%%%%%%%%%%%%%%%%%%%%%%%%%%%%
% \subsection{}

%%%%%%%%%%%%%%%%%%%%%%%%%%%%%%%%%%%%%%%%%%%%%%%%%%%%%%%%%%%%%%%%%%%%%%%%
% \subsubsection{}


%% \renewcommand{\bibsection}{}    % avoid (or change) section heading 
%% \bibliographystyle{apalike}
\bibliographystyle{natbib-stefan}
\bibliography{stefan-literature,stefan-publications}  

\end{document}
